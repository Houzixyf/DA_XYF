\chapter{Zusammenfassung und Ausblick}
\label{ch:Zusammenfassung_und_Ausblick}
\section{Zusammenfassung}
Das Ziel der vorliegenden Diplomarbeit war die Analyse der Überführungstrajektorien eines unteraktuierten Systems aus der Umgebung einer Ruhelage in diese Ruhelage mittels des Python-Pakets \emph{PyTrajectory}, welches um eine Funktion der Bestimmung der optimalen Überführungszeit erweitert wurde. 

Das Theorem von Brockett zur Bestimmung der Existenz einer stetigen differenzierbaren Regelgesetz wurde zuerst in Kapitel \ref{ch:Brockett_Bedingung_für_unteraktuiertes_mechanisches_System} vorgestellt. Als Anwendung existiert für das in Kapitel \ref{ch:Trajektorieplanung_durch_Lösung_einer_Randwertaufgabe} und \ref{ch:Reglerentwurf-durch-Trajektorieplanung} betrachtete Beispiel-das nicht-holonomen Doppelintegrator-System nach Brocketts Theorem keines stetig differenzierbares Regelgesetz zu verfügen, um die Ruhelage asymptotisch zu stabilisieren. Im Gegensatz dazu erfüllt das berühmte Benchmark System-``inverse-Pendel'' diese notwendige Bedingung.

Zum Vergleich der Überführungstrajektorien dieser zwei Systeme wurde das Python-Paket \emph{PyTrajectory} als das Werkzeug verwendet, um die Trajektorien durch Lösung einer Randwertaufgabe zu entwerfen. Eine Funktion davon wurde erweitert, damit die Überführungszeit des Regelvorgangs auch ein Berechnungsergebnis ist, statt vom Nutzer vorgegeben zu werden. Der Ausgangspunkt lag in der Transformation der Zeitkoordinaten, die durch die Ergänzung eines zusätzlichen Parameters in der Systemzustandsfunktion realisiert wurde. Aber wegen des nicht brauchbaren Ergebnis des Parameters wurde eine Straffunktion entworfen, mit der eine bessere Lösung ausgerechnet wurde.

Danach wurde die Überführugstrajektorie zweier Systeme mit unterschiedlichen Anfangswerten von Systemzuständen mittels \emph{PyTrajectory} analysiert. Die Regelmäßigkeit der Trajektorien ist nur in einem \textbf{begrenzten} Bereich für beide Systeme effektiv hängt nicht davon, ob das System die Brockett-Bedingung bezüglich erfüllt. Mit anderen Worten ist die Konstruktion eines Regelgesetz zur Lösung der Randwertaufgabe mit Anfangszuständen in einer relativ großen Umgebung der Ruhelage schwierig. Aus diesem Grund wurden in Kapitel \ref{ch:Reglerentwurf-durch-Trajektorieplanung} einige nicht stetig-differenzierbaren Regelgesetze unter Berücksichtigung der Besonderheit der Systemzustandsdarstellung für den nicht-holonomen Doppelintegrator entworfen, mit den die Trajektorien aus der Umgebung der Ruhelage in diese ermöglicht wurden. 
\section{Ausblick}
\label{Ausblick}
Für die Zukunft empfiehlt die Autorin die Vervollständigung der Analyse der Überführungstrajektorien aus einer Umgebung der gewünschten Ruhelage in diese. In der Arbeit wurden nur drei typischen unteraktuierte Systeme untersucht. Das Modell wie das Acrobot-System \cite{spong1995swing} wäre beispielsweise ein guter Anfangspunkt.

Aus Zeitgründen erwägt die Autorin in der Arbeit den Entwurf eines Steuergesetzes durch die Interpolation der ausgerechneten Trajektorien aus $\emph{PyTrajectory}$, das nur in einem sehr einfachen Fall verwendet wurde, wobei sich nur ein Systemzustandselement in einem Bereich veränderte. Die Konstruktion eines Regelgesetz zur Lösung der Randwertaufgabe mit den Anfangswerten von $\vect{x}$ in zweidimensionalen Umgebung der Ruhelage soll auch durch Interpolation bekannter Trajektorien umsetzbarmöglichst geringer Bogenlängemöglichst geringer Bogenlänge sein. 
